\section{Recommendations}
As seen in previous sections decent results were already accomplished, however many possibilities for improvements are readily found and much more can be done in this area. First of all, many physical improvements could be made to the testbed. Moreover, the signal analysis part of this research could be improved. Last, further steps can be taken towards implementation of this research into autonomous or assisting driver system.

\subsection{Testbed}
As mentioned previously, the current testbed leaves room for improvement. The car is showing signs of wear and multiple impact related accidents have taken its toll on the frame. As it is, the frame is now warped and requires an offset on the front wheels to drive straight again at lower power ranges. However, when driving in the high power ranges the vehicle dynamics become different and the offset will eventually make the car turn. Although this offset can be corrected in the measurements, the fact that it is necessary suggests there are other problems with the data being generated.

Moreover, the measurements are affected by vibrations. To fit the testbed to the Bicycle Model, the original suspension has been replaced by stiff turnbuckle rods, that do not dampen any vibrations of the car. The IMU, unfortunately, is highly susceptible to this noise. Reducing vibrations should be high on the priority list when attempting to improve the results of this research.

Finally, the 3D-printed wheels are hard to balance as the tolerance on the construction method is quite high. As a result of this, the car shows excessive vibrations during tests. This is further enlarged by placing the magnets which may have even higher tolerances. Some effort to improve results might be aimed towards building proper balanced wheels using more precise techniques as a lathe, wheel balancing tools and maybe even custom magnets.

\subsection{Signal analysis}
Another area of improvement is the signal analysis. During this research an attempt was made to convert the signal of the IMU into the Fourier spectrum. However, it was difficult to distinguish the spectrum of noise from the actual signal. Therefore, a filter was created with a cut-off frequency of 5 Hz. It was determined that vibrations from the wheels would be generated from 5.5 Hz onwards and therefore the 5 Hz cut-off frequency was applied. Further signal analysis might improve on this rudimentary constructed filter. When combined with noise reduction, the signal might be cleaned up significantly to give a more clear Fourier spectrum on the noise and the actual signal. By this, a reduced filter can be designed, which might generate a better picture through more data points.  
\subsection{Implementation}
Finally, we look at implementation. The eventual aim of this research is to be able to use it in Nonlinear Model Predictive Control and eventually implementation into cars. However, when looking towards the future it is evident a couple of problems will arise. The first problem with a tire model is that it is heavily dependent on more variables than initially taken into account in this research. A quick grasp would be environment variables like temperature, terrain and humidity. Although these might be determined by adding more sensors to the vehicle, there are also variables which are more difficult to determine. A way of determining the model at any given instant of time might be a more viable approach. This raises new problems though. To determine data points throughout the entire spectrum, the car needs to gather data at high slip ratios at that given time. This is something which can not always be done easily. Further research into generating a model without a lot of especially high slip ratio data points needs to be done to make this possible.