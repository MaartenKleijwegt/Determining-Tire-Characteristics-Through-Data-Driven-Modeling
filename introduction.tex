\section{Introduction}
In the United States alone, the National Highway Traffic Safety Administration \cite{NHTSA} reports that traffic accidents have taken 37,000 lives in 2016. Until now, driver safety systems in modern cars have focused on keeping the friction of the tires of cars in the linear regime while driving. For example, ABS prevents the wheels from locking and traction control prevents the car from losing grip on the road. However, when we look towards professional rally drivers, we see that operating in the linear regime is not specifically necessary for driving safely. If cars were to operate safely in the nonlinear regime of friction, more accidents can be prevented. It would give Driver Aid Systems more capabilities and options. This might result in better accident prevention solutions.

This project focuses on establishing a measurement setup and determining the tire characteristics of a small-scale RC car with it. It is a step towards developing a correct Nonlinear Model Predictive Control (NMPC) system that enables autonomous driving on the limits of handling. NMPC is a control method which uses a dynamical model that can predict the motion of the car. Therefore, this system is able to let the car make evasive maneuvers up to and including the nonlinear regime of driving \cite{Tamas}. In order to obtain a correct dynamical model for the NMPC, tire characteristics are required.

There are a few ways to obtain these tire characteristics. Usually, they are determined using special test rigs in which only the wheels are placed. However, these test rigs are very expensive and the characteristics are only valid for the given tire, in only the given condition (think of temperature, state and wetness of the tire) on only a single given surface \cite{Jonson}. A lot of testing can be done to determine the characteristics of all tires in all conditions. Yet it could be far better if a car would be able to determine its own characteristics using on board sensors such as GPS and Inertial Measurement Units (IMU). Testing with real cars and on-board sensors in a special test environment is a possibility, but developing and refining this technique on small scale RC cars first requires less resources such as large test tracks and expensive cars and tires. Therefore, using a small-scale RC car equipped with sensors is by far the least expensive way to obtain these tire characteristics. When proven successful, these methods can then be applied on actual cars to determine their tire characteristics.

This paper strives to determine the tire characteristics of a small scale RC car using only on board sensors and a motion capture system that represents GPS-type information. Section 2 discusses some scientific background on the topic. After this, the experimental setup and data processing will be explained in Section 3. Next, in Section 4, results will be discussed. Finally, the conclusions will be given in Section 5. The paper also leaves some room for recommendations and acknowledgements.  

